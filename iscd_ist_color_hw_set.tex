
\documentclass{amsart}

\newtheorem{theorem}{Theorem}[section]
\newtheorem{lemma}[theorem]{Lemma}

\theoremstyle{definition}
\newtheorem{definition}[theorem]{Definition}
\newtheorem{example}[theorem]{Example}
\newtheorem{xca}[theorem]{Exercise}

\theoremstyle{remark}
\newtheorem{remark}[theorem]{Remark}

\numberwithin{equation}{section}

%    Absolute value notation
\newcommand{\abs}[1]{\lvert#1\rvert}

%    Blank box placeholder for figures (to avoid requiring any
%    particular graphics capabilities for printing this document).
\newcommand{\blankbox}[2]{%
  \parbox{\columnwidth}{\centering
%    Set fboxsep to 0 so that the actual size of the box will match the
%    given measurements more closely.
    \setlength{\fboxsep}{0pt}%
    \fbox{\raisebox{0pt}[#2]{\hspace{#1}}}%
  }%
}

\begin{document}

\title{ Digital Color Management Q \& A }

\author{Bruce B. Campbell}

\address{}


\dedicatory{This paper is dedicated to our supporters.}

%\keywords{Bayesian Compositing}

%\begin{abstract}
%\end{abstract}

\maketitle


\begin{enumerate}
\item Light sources are characterized by their spectral power
distributions.  The spectral power distribution $\rho(\lambda)$ is
the fraction of the total power emitted from a source at
wavelength $\lambda$. \item Daylight is a mixture of direct
sunlight, and light that is scattered and diffracted by the
atmosphere (skylight).  The power distribution of daylight changes
according to weather, time of day, and atmospheric contamination.
\item For the purposes of color measurement, objects are
characterized by their spectral reflectance $R(\lambda)$ which is
the the fraction of incident light at wavelength $\lambda$ that is
reflected from a point on the object. \item A green object will
not always appear to be green.  This can happen under the
following circumstances;
\begin{itemize} \item if the spectral power distribution
of the light source has low power in the same are of the spectrum
where the spectral reflectance is high, then the object will
appear to be different from green. \item if the object is
fluorescent and is exposed to a light source that excites the
particular wavelength that causes a shift in the reflectance
distribution away from green.  \item colorblind observer will not
detect a green object to be green.\end{itemize} \item A color
stimulus is a color of light to detected by some observational
means.  The color stimulus is characterized by conditions of the
light and the objects being observed.  The stimulus at wavelength
$\lambda$ is the product of the spectral power distribution of the
light source and the spectral reflectance of the object;
$S=\rho(\lambda)*R(\lambda)$ \item The human audio and visual
systems are fundamentally different in terms of frequency domain
processing and the spatial resolution capability. Spatial location
of a stimulus is much better with the visual system.  The audio
system is capable of decoding and resolving any waveform with
frequency components less than about 26kHz using a single
detector.  The human visual system uses three different receptors
for light. These detectors have some overlap in their
sensitivities - the overlap in the sensitivity gives some degree
of freedom to the resolving process. It's underdetermined, so
there are multiple inverse solutions to a particular measurement
of a stimulus - $M(S)$ - to the human visual system. This property
is called metamerism.  An imaging system may be able to resolve
stimuli that the human visual system can distinguish, this can
complicate output if not handled. Metamerism simplifies imaging
systems because it reduces the number of colors that must be
faithfully reproduced.  Some imaging systems get by with 256
colors - down from the 11 million that can be detected by the
human visual system.  Think of this as a lossy
compression/decompression. \item Objects may not appear the same
under differing light sources because of metamerism.  The stimuli
might be the same for both under one light source even though the
reflectance differs. Changing the light source changes
$\rho(\lambda)$ which may elicit different stimulus responses
$S$for the two objects. \item Colorimetry is the process of
encoding and measuring colors for retrieval and display in imaging
systems.

\item  True or false: The spectral responses of the CIE standard
observer are identical to the cone responses of the average human
observer. "   Why, or why not? \par  False.  The CIE trichromatic
color matching functions are a linear combination of any average
cone response functions.

\item What information is needed to compute CIE tristimulus
values? \par The spectral power distribution of the light source
\(S(\lambda)\), the spectral reflectance of the
object,\(R(\lambda)\), the CIE color matching functions;
\begin{math}\overline{x}(\lambda),\overline{y}(\lambda),\overline{z}(\lambda)
\end{math}, and a normalizing constant are integrated over \(
\lambda \) in \( [380,780] \).
\begin{eqnarray}
 \nonumber X = \int R(\lambda) S(\lambda) \overline{x}(\lambda) d\lambda \\
 \nonumber Y = \int R(\lambda) S(\lambda) \overline{y}(\lambda)
d\lambda \\ \nonumber Z = \int R(\lambda) S(\lambda)
\overline{z}(\lambda) d\lambda
\end{eqnarray}

\item Would it be possible to build an output device, such as a
video projector, using the primaries defined in the CIE XYZ
system? " Why, or why not? \par No.  The primaries in the CIE XYZ
system were chosen to give the color matching function
non-negative values.  This requires the primaries to have negative
power in some region of the spectrum, which can't really be
realized in a light source.

\item What does a CIE x, y chromaticity diagram show? "   What
does it not show? \par If the $X, Y, X$  tristimulus values are
normalized , the plot of $X/(X+Y+Z), Y/(X+Y+Z)$ gives the
qualities of a color stimulus with the luminance normalized out.

\item True or false: The appearance of a color can be described by
a set of CIE X, Y, Z tristimulus values.\par False.  The CIE
coordinate system is useful for describing color differences for
small differences in stimuli, but not for describing the
appearance of a color.

\item True or false: The appearance of a color can be described by
transforming its CIE X, Y, Z tristimulus values to CIELAB or
CIELUV color space.
\par  False.   The appearance of a color is dependent on the
viewing conditions.  CIELAB and CIELUV color spaces are good for
measuring differences in color.

\item  Describe an appropriate use for a Status A densitometer.
Describe an inappropriate use for a Status A densitometer.
\par  An appropriate use of a Status A densiometer would be to measure or to monitor
the optical density of the output of a three channel imaging
system.  An inappropriate use would be to do colrimetry
calculations base on desiometer readings.  The Status A
densiometer response functions are tuned to the narrow bandwidth
supported by dyes.  Also using the Status A readings to compare
output from different imaging systems is not advised.  The Status
A readings from different systems may agree, but that does not
imply the systems generate images with the same appearance.

\item What are the three basic functions required in all imaging
systems?
 · Define each of these functions.
\par Capture, signal processing, and image formation.  Capture is
the process of detecting light stimuli from a scene.  Signal
processing is modifies the captured image for output.  Image
formation takes the processed signal and uses it to control the
color forming elements of the output device.

\item Why is color separation trichromatic in most color-imaging
systems?
 · Describe an application in which color separation other than trichromatic might be required.
 \par Most color imaging systems are trichromatic because the
 human visual system is.  X-ray capture is monochromatic, and some landsat imaging imaging systems
 may have more than three bands of capture.

\item How can color separation be accomplished in an electronic
camera?
\par Color separation in an electronic camera may be achieved by
capturing image data on a CCD mosaic of RGB sensors.  Light can
also be optically separated into three bands and then sent to
three different CCD sensors - one for each band.

\item How is color separation accomplished in photographic films?
\par Color separation in photographic film is done through three
or more light sensitive layers and filters.

\item What is meant by exposure factor?
 · How can exposure-factor values be calculated?
\par The exposure is calculated by integrating the spectral
responsivity with the light spectrum and the spectral reflectance
or transmittance of the object being imaged.

\item What is the basic function of color signal processing?
 · Describe some of the transformations typically included in color signal processing.
\par The basic function of color signal processing is to make an
image suitable for viewing.  Typical functions are signal
amplification, non-linear modification of the neutral signal, and
color matrix calculations, sharpening and noise reduction, and
possibly compression.

\item What are the two basic types of color image formation? · Are
other types of color image formation possible? · If so, describe
at least one.
\par The two basic types of color image formation are additive
color mixing and subtractive color mixing.  Other types are
possible.

\item In an additive system, what color is formed by: · Red and
blue? · Blue and green? · Red and green? · Red plus green and
blue?
\par In an additive color forming system, red and blue make
magenta, blue and green make cyan, red and green make yellow, and
red plus blue plus green make white.

\item In a subtractive system, what color is formed by: · Cyan and
yellow? · Yellow and magenta? · Magenta and cyan? · Cyan plus
magenta and yellow?
\par In a subtractive color forming system, cyan and yellow make
green, yellow and magenta make red, magenta and cyan make blue,
and cyan plus magenta plus yellow make black.

\item Define a complete color-imaging system.
 · Is a digital camera a complete color-imaging system? Why, or why not?
 · Is a photographic slide film a complete color-imaging system? Why, or why not?
 \par A complete color imaging system is one that can perform
 capture, signal processing, and image formation.  A digital
 camera is such a system.  Capture and signal processing take
 place in the camera, and image formation takes place on the LCD
 screen on the back.  Photographic slide film is a complete color
 imaging systems well.  Capture takes place in the camera,
 signal processing and image formation takes place in the lab.

\item Describe an application in which an instrument, such as a
colorimeter, would be preferred for assessing color quality.
\par A colorimeter would be used to compare an original scene to a
reproduction produced by a color imaging system.

\item Describe an application in which a human observer would be
preferred for assessing color quality.
\par Taking a picture with photographic film, and then viewing the
reproduction later.

\item What aspect of the human visual system corresponds to:
 · image capture
 · signal processing
 · image formation
\par Image capture takes place in the rods and cones of the eye,
signal processing and image formation both take place in the
brain.

\item What is psychological signal processing?
\par Psychological signal processing is the component which
accounts for color memory, and color preference.

 \item What is psychophysical signal processing?
\par Psychophysical signal processing is the component of image
processing in the human brain which takes into account effects
like chromatic adaptation, lateral brightness adaptation, and the
relative nature of luminance detection.

\item. Define each of the following:
 · general-brightness adaptation
 · lateral-brightness adaptation
 · chromatic adaptation
 \par Brightness adaptation is how the eye responds to varying
 levels of illumination.  Lateral brighness adaptation is how the
 eye has different sensitivities in different parts of the eye.
 Chromatic adaptation is how the eye responds to an average chromatic
 stimulus of a scene.

\item For each adaptation effect listed in question 6, give an
example of how the effect might be encountered in a practical
imaging situation.
\par Brightness adaptation would be encountered when viweing
conditions change from low to high light illumination.  Lateral
brightness adaptation would be encountered when focusing on
different objects - closer objects are captured with the fovea.
Chromatic adaptation occurs in incandecent illumination induces an
increase in short wavelenght sensitivity.

\item Describe the relationship of CIE colorimetry to human color
vision.
 · What aspects of color vision are predicted well by standard CIE colorimetry?
 · What aspects are not predicted well? Why not?
\par CIE colorimetry can be used to predict if two stimuli will
visually match under identical viewing conditions.  CIE
colorimetry can not be used to emulate signal processing of color
formation functions.

\item For the purposes of color characterization, what
characteristics of a color monitor must be measured or otherwise
determined?

\par The CIE x,y chromaticity coordinates of the monitor red, green, and
blue primaries must be determined experimentally, or be given by
the manufacturer.  The location of the chromaticity coordinates
determines the gamut of colors the monitor can produce by adding
varying intensities of the red, green, and blue primaries. The
color matching functions of the CIE standard observer can be
obtained from the x,y chromaticity coordinates of the primaries.

\item What does “monitor white point” mean?

\par The monitor white point is the chromaticity coordinates
of the stimulus obtained by mixing the red, green, and blue
primaries at full intensity - $ (R,G,B)=(255,255,255) $ for an 8
bit monitor.

\item What does “monitor grayscale tracking” mean?
 · Why is such tracking important?
 \par Monitor greyscale tracking is the ability to emit light of
 constant chromaticity but varying luminance when $R=G=B$.

\item Monitor and hardcopy images are said to be “highly
metameric”.
 · What does this mean?
 · Describe a practical consequence of a high degree of metamerism.
 · Describe a method for dealing with that consequence.
\par For a monitor to reproduce a color stimulus with a monitor we add light
from three sources.    Metamerism is when the spectral power
distribution of the light source produced by the monitor is very
different from the spectral power distribution of the stimulus -
even though the CIE tristimulus values of the two stimuli are the
same.  The spectral power distribution of the three sources
determine the amount of each light source needed to produce a
particular stimulus.  For hard copy the spectral reflection
density of the dye set used for the print is what determines the
amount of each dye to use for a given stimulus.

Because of the differences among individual responsivities differ
from the CIE standard observer, two individuals observing a color
on the monitor may disagree whether the stimulus matches a
reference one.  Color matching experiments can be done for each
observe to correct for this.

\item Why should (or should not) logarithmic units be used in
characterizing the grayscales of monitors?

\par Taking the log of monitor luminance gives a better measure of differences at the low end of the luminance scale.
The CIE L-a-b luminance is defined as  $L^{*}=116
(\frac{Y}{Y_n})^{\frac{1}{3}} -16 $.  Conveniently, the ...


\item What grayscale characteristic must a video camera have for
the entire system to have a grayscale that is one-to-one with that
of the original scene?

\par The greyscale characteristic of the camera needs to be the inverse
of the greyscale characteristic of the monitor.

\item  In addition to an appropriate grayscale characteristic,
what further characteristic must a video camera have for the
entire system to have colorimetric color reproduction that is
one-to-one with that of the original scene?
\newline
\par The chromaticities of the non-neutral scene colors must match
the output device chromaticities.
\newline
\item Describe the appearance of such colorimetric color
reproduction.
\newline
\par When the chromaticities of the input match those of the output
device, the result is something the viewer would report as being
less saturated than the original scene. This is because using the
CIE standard observer color matching functions as camera
sensitivities leads to chromatic errors. Since the CIE Standard
Observer color matching functions are a linear transform of any
other color matching functions, we can remove these errors by
transforming the camera color matching function from the ones
representing the monitor primaries to another set that accurately
reproduces the color stimuli.
\newline

 \item  What additional factors must be taken into
account in order to produce pleasing reproductions of original
scenes?
\newline
\par In addition to accurate colorimetric reproduction,
psychophysical effects of the human visual system must be taken
into account in order to produce visually pleasing results.
\newline

\item  How can video signal processing be used to account for
these factors?
\newline
\par The greyscale characteristic can be modified to account for
viewing flare.  Also, the reproduction device is of a lower
overall luminance than the original scene - which leads to a
reduction in the perceived saturation.  Boosting the greyscale
characteristic accounts for this effect as well.
\newline

\item Describe the principal components of an ideal video system.
 · How do practical systems correspond to this ideal?
 · Why will the color gamut of any real CRT be limited?
\newline
\par An ideal video system would have camera spectral
sensitivities with all positive color matching functions so that
all color could be realized by the monitor.  There should be a
color matrix transformation that converts the camera color
matching functions to ones that correspond to the monitors.  The
gamut of a monitor will be limited because the transform from the
CIE standard observe to monitor color matching functions will give
colors with negative amounts of monitor RGB primaries.  The CIE
standard observer color matching functions were all positive, and
all colors could be reproduced because the primaries were
imaginary.  The monitor primaries are not imaginary - this is the
reason why the color matching functions of the monitor must have
negative values.
\newline

\item Describe three possible methods for encoding video signals.
\newline
\par We could just do the colorimetric matching.  We could do the
color matching and then let Ed tweak the signal processing so the
results are pleasing.  Lastly we could reverse engineer any
imaging system to produce pleasing results - get data on the
output, and taylor the signal processing to produce the
characteristics the viewers have reported as pleasing.  This last
suggestion works for any given imaging system, but we would not be
able to design the signal processing of an input or output device
in isolation this way.

\item  What general characteristics are shared by virtually all
reflection media?
\newline
\par  Most reflection-print media consist of three or more
subtractive dyes used to form an image on some type of reflective
support material.
\newline
\item  What happens when a ray of light strikes the surface of a
simple reflection medium?
\newline
\par Some incident light will be reflected on the surface of the media,
the rest will be refracted through the colorant layer and
scattered off the reflective support. Some light scattered off the
support will be reflected off the surface coating back to the
colorant. 
\newline

 \item  A neutral object is
photographed under a first illuminant. A reproduction is made and
viewed under a second illuminant. For the reproduction of the
object to appear neutral, should its chromaticity match that of
the first illuminant or that of the second illuminant? Why?
\newline
\par
The chromaticity of the reproduction neutral should be the the
chromaticity of the illuminant the reproduction is viewed in.  The
observer will be chromatically adapted to the second illuminant
when viewing the reproduction.  If the chromaticity of neutral
objects are reproduced with the chromaticity of the first
illuminant, the observer will perceive the difference.
\newline

\item  What is meant by “viewing illuminant sensitivity”? Why it
is important in imaging applications?
\newline
\par Viewing illuminant sensitivity is a measure of how much image dye components must be modified to produce a
metamer when reproducing neutrals to be viewed under different
illuminants.   This is important in imaging systems because the
same reproduction could be viewed under different illuminants, and
the observer may perceive a difference under the two illuminants
if the viewing illuminant sensitivity is high.
\newline

\item Why would the grayscale characteristic of a reflection-print
system not be one-to-one with the original scene?
\newline
\par If the viewing conditions of the scene are different from the viewing
conditions of the reproduction, the grayscale would need to be
modified.  The contrast is boosted for the midrange and rolled off
at the ends of the density range.  This is done to accommodate for
viewing flare and differences in the absolute luminance of the
scene and the viewing environment of the reproduction.

\item  System A has a higher maximum density and a lower minimum
density than System B. Which system will require a higher
mid-scale gamma?
\newline
\par The dynamic range of system A is greater than system B.  The
midscale gamma of system B will need to be higher to produce dark
and light tones at acceptable density.

\item  If the taking illuminant and viewing illuminant of a
reflection-print system differ in chromaticity, what is required
in order to make a meaningful evaluation of reproduced
chromaticity values?
\newline
\par The observes state of chromatic adaptation in the viewing environment must be taken into
account.
\newline

\item  Why might the reproduced colorimetry of a reflection-print
system differ from that of an original scene?
\newline
\par To accommodate for a difference viewing
illuminant and scene illuminant, the chromaticities could linearly
transformed to produce a visual match for the tristimulus values
of the original color stimulus.
\newline

\item  Is there a single best color-reproduction position that all
reflection-print systems should attempt to realize? Why, or why
not?
\newline
\par  The color reproduction of reflection print systems is
dependent on differences in scene and viewing conditions and the
adaptive state of the observer.  There is no common method to
account for all the different possibilities for scene and
reproduction viewing environment.

\item What properties make photographic transparencies popular for
use as input to color-imaging systems?
\newline \par High dynamic range, and narrow spectral
sensitivities make slide films popular inputs to high performance
imaging systems.  Photographic transparencies are popular in the
graphics arts and advertising industries since the image can be
viewed directly on the media.
\newline

\item  How do the relative red, green, and blue speeds of a
tungsten-balanced photographic transparency film compare to those
of a daylight-balanced photographic transparency film? Why?
\newline \par Tungston light sources have more long wavelength
power than $D_{55}$.  To compensate for this, the spectral
sensitivities of the tungsten balanced films have lower
sensitivity in the long wavelengths and higher sensitivity in the
short wavelengths.  The bandwidths of the three channels are the
same for tungsten and $D_{55}$ balanced film.
\newline

\item  What does it mean to say that one dye is purer than
another?
\newline \par If the bandwidth - or area of the spectrum a dye is
sensitive to - of one dye is narrower than another, we say that
dye is more pure.
\newline

\item   If the same magenta dye is coated on a reflective and a
transmissive support, how will the spectral properties of the
resulting coatings differ? Describe one consequence of this effect
in practical imaging applications.
\newline \par On a reflective media, the light passes through each dye layer at least twice
before reaching the viewer.  This allows for more absorption of
light.  The effective density of the magenta dye on the reflective
support will be higher than the density on the transmissive media.
On the reflective support, there will be more absorption of light
at the boundary of the band the dye is sensitive to.  This leads
to a change in the chromaticity of the reproduced colors since the
magenta dye on the reflective support will absorb more blue and
red light than the same dye on a transmissive media.  More color
processing is required for the reflective media to correct for
this shift.
\newline

\item   Why is viewing-illuminant sensitivity of lesser importance
for transmission media than it is for reflection media?
\newline \par Transmissive media are designed for both scene and
viewing illuminants.  Slide films and photographic transparancies
are designed to be viewed under a known illumination - tungston
for slide film and $D{55}$. Since the viewing illuminant is known
ahead of time, the dyes can be tuned in ways that reflection media
can't. More pure dyes which give a larger gamut of reproduced
colors can be used.  Note this does make transmissive more
illuminant sensitive than reflection media - we could expect more
color shifts in viewing transparant media with the wrong light
source.
\newline

\item   Why are 35mm slide films balanced cyan-blue, according to
instrument measurements?
\newline \par To compensate for the incomplete chromatic adaptaion
to the viewing illiminant that takes place when viewing slide
images in a darkened room.  The viewer does not completely adapt
to the tungsten light source, so a colroimetric neutral would
appear to have extra red light.  To compensate for this extra cyan
dye is added to absorb the unwanted red light.
\newline

\item  Name two other ways in which the grayscale characteristic
of a photographic transparency film differs from that of
reflection-print system. Why do these differences exist?
\newline \par The dynamic range ($D_{max}/D_{min}$) of
transparency film is higher than reflection media and the gamma is
higher.  The dynamic range is higher because of the more pure
dyeset used in transmissive media.  The gamma is higher to account
for lateral brighness adaptation in the viewing environment.  The
dark surround of the slide image causes a perceived drop in
contrast which is accounted for by increasing the gamma.
\newline

\item  Why are the spectral sensitivities of photographic
transparency films significantly different from any set of
color-matching functions?
\newline \par Since the chemistry of the transparancy system is
relatively fixed, the non-linear signal processing is built into
the spectral sensitivities of the dyes used.
\newline

\item   Why are image overall density and color balance somewhat
less critical for photographic transparency films compared to
reflection-print systems? How does this affect the color encoding
of these films?
\newline \par Density and color balance are less important for
transparancy film because the image is driving the viewers state
of adaption. This allows the brain to do signal processing that
has to be built into reflection media. 
\newline

\item   Describe at least three other complications of color
encoding of photographic transparency films.
\newline \par The relationship between colorimetry and color
perception is sensitive to viweing illmuninant.  When scanning
transparancies, the illuminant of the scanner determines the
colors reproduced.  If the scanner illuminant is different from
that designed into the transparancy, the image will need to be
color corrected.

\item  What properties make photographic negatives well suited for
use as input to color-imaging systems?
\newline \par The large dynamic range and low gamma of
photographic negatives give scanners the ability to capture images
covering a wide range of exposures with lower sensitivity sensors.
\newline
\item What is the major complication in using photographic
negatives as input to color-imaging systems?
\newline \par Color negatives are not designed for human viewing
so any imaging application that requires previewing images prior
to output will require additional processing to present the images
in human viewable form.  Until recently scanning of negatives was
slow and costly so previewing negatives required the viewer to be
able to interpret negative images, or a set of positive images
printed from the negatives.
\newline
\item  What is the fundamental difference between a
photographic negative system and photographic transparency system?
\newline \par Maximum image dye is formed at minimum exposure for
a negative and maximum exposure for a transparency.
\newline
\item  What does “printing density” mean?
\newline \par Printing density is the negative log of the
R,G,B exposures for the photographic media used to record a
negative image.
\newline
\item  Why is it important in the measurement of photographic
negatives?
\newline \par Greyscale and color reproduction of photographic
negatives are designed around the reflectances of the printing
dyes used.
\newline
\item  How can printing density
values be determined, using an ISO Status M densitometer for
measurement?
\newline \par Apply a color matrix to the signal.
\newline
\item  What is “printing
density metamerism”?
\newline \par Print density metamerism occurs when two negative signals have different spetral
transmission responses but produce the same optical printing
densities.
 \newline
\item  Why are the
red, green, and blue printing-density gammas matched for
photographic negative films? What would happen if the were not
matched?
\newline \par R,G,B print densities are parallel so netural
tonescales are accurately reproduced when the negatives are
printed.
\newline
\item  What is the optimum gamma
for a photographic negative film?
\newline \par It depends on the application, indoor portrate film
has a lower gamma than consumer film.
\newline
\item What is the most difficult problem encountered in color
encoding images from photographic negatives?
\newline \par Print density metamerism.

\item List some reasons why matrix correction may be required in a
color imaging system.
\newline \par Transform color signal to a new color space.
\newline
\item What is “printing density cross talk”? How can such cross
talk be corrected in optical-printing systems? How can cross talk
be corrected in digital-printing systems?
\newline \par When two dyes have overlapping spectral
sensitivities.
\newline
\item Predict the effect that increasing right-way blue-onto-green
color correction will have on these colors: neutrals, reds,
greens, blues, cyans, magentas, yellows, and skin tones.
\end{enumerate}


\end{document}

%------------------------------------------------------------------------------
% End of journal.tex
%------------------------------------------------------------------------------
